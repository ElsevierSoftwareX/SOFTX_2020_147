\section{Program gto\char`_segment}
The \texttt{gto\char`_segment} segments a filtered sequence.\\
For help type:
\begin{lstlisting}
./gto_segment -h
\end{lstlisting}
In the following subsections, we explain the input and output paramters.

\subsection*{Input parameters}

The \texttt{gto\char`_segment} program needs two streams for the computation, namely the input and output standard. The input stream is a numeric file.\\
The attribution is given according to:
\begin{lstlisting}
Usage: ./gto_segment [options] [[--] args]
   or: ./gto_segment [options]

It segments a filtered sequence.

    -h, --help                show this help message and exit

Basic options
    -t, --threshold=<dbl>     The segment threshold
    < input.num               Input numeric file (stdin)
    > output                  Output the segment file (stdout)

Example: ./gto_segment -t <threshold> < input.num > output
\end{lstlisting}
An example on such an input file is:
\begin{lstlisting}
1   1.77
5   2.18
10  2.32
15  3.15
20  2.52
25  4.43
30  1.23
\end{lstlisting}

\subsection*{Output}
The output of the \texttt{gto\char`_segment} program is the interval of values ​​below the threshold.\\
An example, for the input using a thresould of 3, is:
\begin{lstlisting}
0:10
\end{lstlisting}