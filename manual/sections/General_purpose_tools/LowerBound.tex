\section{Program gto\char`_lower\char`_bound}
The \texttt{gto\char`_lower\char`_bound} sets an lower bound in a file with a value per line.\\
For help type:
\begin{lstlisting}
./gto_lower_bound -h
\end{lstlisting}
In the following subsections, we explain the input and output paramters.

\subsection*{Input parameters}

The \texttt{gto\char`_lower\char`_bound} program needs two streams for the computation, namely the input and output standard. The input stream is a numeric file.\\
The attribution is given according to:
\begin{lstlisting}
Usage: ./gto_lower_bound [options] [[--] args]
   or: ./gto_lower_bound [options]

It sets an lower bound in a file with a value per line.

    -h, --help                show this help message and exit

Basic options
    -l, --lowerbound=<int>    The lower bound value
    < input.num               Input numeric file (stdin)
    > output.num              Output numeric file (stdout)

Example: ./gto_lower_bound -l <lowerbound> < input.num > output.num
\end{lstlisting}
An example of such an input file is:
\begin{lstlisting}
0.123
3.432
2.341
1.323
7.538
4.122
0.242
0.654
5.633
\end{lstlisting}

\subsection*{Output}
The output of the \texttt{gto\char`_lower\char`_bound} program is a set of numbers truncated at the a defined lower bound.\\
Using the input above, an output example for this is the following:
\begin{lstlisting}
Using lower bound: 2
2.000000
3.432000
2.341000
2.000000
7.538000
4.122000
2.000000
2.000000
5.633000
\end{lstlisting}