\chapter{General purpose tools}
\label{seq}

\begin{enumerate}

\item \texttt{gto\char`_char\char`_to\char`_line}: it splits a sequence into lines, creating an output sequence which has a char for each line.

\item \texttt{gto\char`_reverse}: it reverses the order of a sequence.

\item \texttt{gto\char`_new\char`_line\char`_on\char`_new\char`_x}: it splits different rows with a new empty row.

\item \texttt{gto\char`_upper\char`_bound}: it sets an upper bound in a file with a value per line.

\item \texttt{gto\char`_lower\char`_bound}: it sets an lower bound in a file with a value per line.

\item \texttt{gto\char`_brute\char`_force\char`_string}: it generates all combinations, line by line, for an inputted alphabet and specific size.

\item \texttt{gto\char`_real\char`_to\char`_binary\char`_with\char`_threshold}: it converts a sequence of real numbers into a binary sequence, given a threshold.

\item \texttt{gto\char`_sum}: it adds decimal values in file, line by line, splitted by spaces or tabs.

\item \texttt{gto\char`_filter}: it filters numerical sequences.

\item \texttt{gto\char`_word\char`_search}: it search for a word in a file.

\item \texttt{gto\char`_permute\char`_by\char`_blocks}: it permutates by block sequence, FASTA and Multi-FASTA files. 

\item \texttt{gto\char`_info}: it gives the basic properties of the file, namely size, cardinality, distribution percentage of the symbols, among others.

\item \texttt{gto\char`_segment}: it segments a filtered sequence.

\end{enumerate}

\section{Program gto\char`_char\char`_to\char`_line}
The \texttt{gto\char`_char\char`_to\char`_line} splits a sequence into lines, creating an output sequence which has a char for each line.\\
For help type:
\begin{lstlisting}
./gto_char_to_line -h
\end{lstlisting}
In the following subsections, we explain the input and output paramters.

\subsection*{Input parameters}

The \texttt{gto\char`_char\char`_to\char`_line} program needs two streams for the computation,
namely the input and output standard. The input stream is a sequence file.\\
The attribution is given according to:
\begin{lstlisting}
Usage: ./gto_char_to_line [options] [[--] args]
   or: ./gto_char_to_line [options]

It splits a sequence into lines, creating an output sequence which has a char for each line.

    -h, --help        show this help message and exit

Basic options
    < input.seq       Input sequence file (stdin)
    > output.seq      Output sequence file (stdout)

Example: ./gto_char_to_line < input.seq > output.seq
\end{lstlisting}
An example on such an input file is:
\begin{lstlisting}
ACAAGACGGCCTCCTGCTGCTGCTGCTCTCCGGGGCCACGGCCCTGGAGGGTCCACCGCTGCCCTGCTGCCATTGTCCCC
GGCCCCACCTAAGGAAAAGCAGCCTCCTGACTTTCCTCGCTTGGGCCGAGACAGCGAGCATATGCAGGAAGCGGCAGGAA
GTGGTTTGAGTGGACCTCCGGGCCCCTCATAGGAGAGGAAGCTCGGGAGGTGGCCAGGCGGCAGGAAGCAGGCCAGTGCC
GCGAATCCGCGCGCCGGGACAGAATCTCCTGCAAAGCCCTGCAGGAACTTCTTCTGGAAGACCTTCTCCACCCCCCCAGC
TAAAACCTCACCCATGAATGCTCACGCAAGTTTAATTACAGACCTGAAACAAGATGCCATTGTCCCCCGGCCTCCTGCTG
CTGCTGCTCTCCGGGGCCACGGCCACCGCTGCCCTGCCCCTGGAGGGTGGCCCCACCGGCCGAGACAGCGAGCATATGCA
GGAAGCGGCAGGAATAAGGAAAAGCAGCCTCCTGACTTTCCTCGCTTGGTGGTTTGAGTGGACCTCCCAGGCCAGTGCCG
GGCCCCTCATAGGAGAGGAAGCTCGGGAGGTGGCCAGGCGGCAGGAAGGCGCACCCCCCCAGCAATCCGCGCGCCGGGAC
AGAATGCCCTGCAGGAACTTCTTCTGGAAGACCTTCTCCTCCTGCAAATAAAACCTCACCCATGAATGCTCACGCAAGTT
TAATTACAGACCTGAA
\end{lstlisting}

\subsection*{Output}
The output of the \texttt{gto\char`_char\char`_to\char`_line} program is a group sequence splited by \textbackslash n foreach character.\\
Using the input above, an output example for this is the following:
\begin{lstlisting}
A
C
A
A
G
A
C
G
G
C
C
T
C
C
T
G
C
T
G
C
T
...
\end{lstlisting}
\section{Program gto\char`_reverse}
The \texttt{gto\char`_reverse} reverses the order of a sequence file.\\
For help type:
\begin{lstlisting}
./gto_reverse -h
\end{lstlisting}
In the following subsections, we explain the input and output paramters.

\subsection*{Input parameters}

The \texttt{gto\char`_reverse} program needs two streams for the computation,
namely the input and output standard. The input stream is a sequence file.\\
The attribution is given according to:
\begin{lstlisting}
Usage: ./gto_reverse [options] [[--] args]
   or: ./gto_reverse [options]

It reverses the order of a sequence file.

    -h, --help        show this help message and exit

Basic options
    < input.seq       Input sequence file (stdin)
    > output.seq      Output sequence file (stdout)

Example: ./gto_reverse < input.seq > output.seq
\end{lstlisting}
An example on such an input file is:
\begin{lstlisting}
ACAAGACGGCCTCCTGCTGCTGCTGCTCTCCGGGGCCACGGCCCTGGAGGGTCCACCGCTGCCCTGCTGCCATTGTCCCC
GGCCCCACCTAAGGAAAAGCAGCCTCCTGACTTTCCTCGCTTGGGCCGAGACAGCGAGCATATGCAGGAAGCGGCAGGAA
GTGGTTTGAGTGGACCTCCGGGCCCCTCATAGGAGAGGAAGCTCGGGAGGTGGCCAGGCGGCAGGAAGCAGGCCAGTGCC
GCGAATCCGCGCGCCGGGACAGAATCTCCTGCAAAGCCCTGCAGGAACTTCTTCTGGAAGACCTTCTCCACCCCCCCAGC
TAAAACCTCACCCATGAATGCTCACGCAAGTTTAATTACAGACCTGAAACAAGATGCCATTGTCCCCCGGCCTCCTGCTG
CTGCTGCTCTCCGGGGCCACGGCCACCGCTGCCCTGCCCCTGGAGGGTGGCCCCACCGGCCGAGACAGCGAGCATATGCA
GGAAGCGGCAGGAATAAGGAAAAGCAGCCTCCTGACTTTCCTCGCTTGGTGGTTTGAGTGGACCTCCCAGGCCAGTGCCG
GGCCCCTCATAGGAGAGGAAGCTCGGGAGGTGGCCAGGCGGCAGGAAGGCGCACCCCCCCAGCAATCCGCGCGCCGGGAC
AGAATGCCCTGCAGGAACTTCTTCTGGAAGACCTTCTCCTCCTGCAAATAAAACCTCACCCATGAATGCTCACGCAAGTT
TAATTACAGACCTGAA
\end{lstlisting}

\subsection*{Output}
The output of the \texttt{gto\char`_reverse} program is a group sequence.\\
Using the input above, an output example for this is the following:
\begin{lstlisting}
AAGTCCAGACATTAATTTGAACGCACTCGTAAGTACCCACTCCAAAATAAACGTCCTCCTCTTCCAGAAGGTCTTCTTCA
AGGACGTCCCGTAAGACAGGGCCGCGCGCCTAACGACCCCCCCACGCGGAAGGACGGCGGACCGGTGGAGGGCTCGAAGG
AGAGGATACTCCCCGGGCCGTGACCGGACCCTCCAGGTGAGTTTGGTGGTTCGCTCCTTTCAGTCCTCCGACGAAAAGGA
ATAAGGACGGCGAAGGACGTATACGAGCGACAGAGCCGGCCACCCCGGTGGGAGGTCCCCGTCCCGTCGCCACCGGCACC
GGGGCCTCTCGTCGTCGTCGTCCTCCGGCCCCCTGTTACCGTAGAACAAAGTCCAGACATTAATTTGAACGCACTCGTAA
GTACCCACTCCAAAATCGACCCCCCCACCTCTTCCAGAAGGTCTTCTTCAAGGACGTCCCGAAACGTCCTCTAAGACAGG
GCCGCGCGCCTAAGCGCCGTGACCGGACGAAGGACGGCGGACCGGTGGAGGGCTCGAAGGAGAGGATACTCCCCGGGCCT
CCAGGTGAGTTTGGTGAAGGACGGCGAAGGACGTATACGAGCGACAGAGCCGGGTTCGCTCCTTTCAGTCCTCCGACGAA
AAGGAATCCACCCCGGCCCCTGTTACCGTCGTCCCGTCGCCACCTGGGAGGTCCCGGCACCGGGGCCTCTCGTCGTCGTC
GTCCTCCGGCAGAACA
\end{lstlisting}
\section{Program gto\char`_new\char`_line\char`_on\char`_new\char`_x}
The \texttt{gto\char`_new\char`_line\char`_on\char`_new\char`_x} splits different rows with a new empty row.\\
For help type:
\begin{lstlisting}
./gto_new_line_on_new_x -h
\end{lstlisting}
In the following subsections, we explain the input and output paramters.

\subsection*{Input parameters}

The \texttt{gto\char`_new\char`_line\char`_on\char`_new\char`_x} program needs two streams for the computation, namely the input and output standard. The input stream is a matrix file format with 3 columns.\\
The attribution is given according to:
\begin{lstlisting}
Usage: ./gto_new_line_on_new_x [options] [[--] args]
   or: ./gto_new_line_on_new_x [options]

It splits different rows with a new empty row.

    -h, --help    show this help message and exit

Basic options
    < input       Input file with 3 column matrix format (stdin)
    > output      Output file with 3 column matrix format (stdout)

Example: ./gto_new_line_on_new_x < input > output
\end{lstlisting}
An example of such an input file is:
\begin{lstlisting}
1	2	2
1	2	2
4	4	1
10	12	2
15	15	1
45	47	3
45	47	3
45	47	3
45	47	3
55	55	1
\end{lstlisting}

\subsection*{Output}
The output of the \texttt{gto\char`_new\char`_line\char`_on\char`_new\char`_x} program is a 3 column matrix, with an empty line between different rows.\\
Using the input above, an output example for this is the following:
\begin{lstlisting}
1.000000	2.000000	2.000000
1.000000	2.000000	2.000000

4.000000	4.000000	1.000000

10.000000	12.000000	2.000000

15.000000	15.000000	1.000000

45.000000	47.000000	3.000000
45.000000	47.000000	3.000000
45.000000	47.000000	3.000000
45.000000	47.000000	3.000000

55.000000	55.000000	1.000000
\end{lstlisting}

\section{Program gto\char`_upper\char`_bound}
The \texttt{gto\char`_upper\char`_bound} sets an upper bound in a file with a value per line.\\
For help type:
\begin{lstlisting}
./gto_upper_bound -h
\end{lstlisting}
In the following subsections, we explain the input and output paramters.

\subsection*{Input parameters}

The \texttt{gto\char`_upper\char`_bound} program needs two streams for the computation,
namely the input and output standard. The input stream is a numeric file.\\
The attribution is given according to:
\begin{lstlisting}
Usage: ./gto_upper_bound [options] [[--] args]
   or: ./gto_upper_bound [options]

It sets an upper bound in a file with a value per line.

    -h, --help                show this help message and exit

Basic options
    -u, --upperbound=<int>    The upper bound value
    < input                   Input numeric file (stdin)
    > output                  Output numeric file (stdout)

Example: ./gto_upper_bound -u <upperbound> < input > output
\end{lstlisting}
An example on such an input file is:
\begin{lstlisting}
0.123
3.432
2.341
1.323
7.538
4.122
0.242
0.654
5.633
\end{lstlisting}

\subsection*{Output}
The output of the \texttt{gto\char`_upper\char`_bound} program is a set of numbers truncated at the a defined upper bound.\\
An example, for the input, is:
\begin{lstlisting}
Using upper bound: 4
0.123000
3.432000
2.341000
1.323000
4.000000
4.000000
0.242000
0.654000
4.000000
\end{lstlisting}
\section{Program gto\char`_lower\char`_bound}
The \texttt{gto\char`_lower\char`_bound} sets an lower bound in a file with a value per line.\\
For help type:
\begin{lstlisting}
./gto_lower_bound -h
\end{lstlisting}
In the following subsections, we explain the input and output paramters.

\subsection*{Input parameters}

The \texttt{gto\char`_lower\char`_bound} program needs two streams for the computation,
namely the input and output standard. The input stream is a numeric file.\\
The attribution is given according to:
\begin{lstlisting}
Usage: ./gto_lower_bound [options] [[--] args]
   or: ./gto_lower_bound [options]

It sets an lower bound in a file with a value per line.

    -h, --help                show this help message and exit

Basic options
    -u, --lowerbound=<int>    The lower bound value
    < input                   Input numeric file (stdin)
    > output                  Output numeric file (stdout)

Example: ./gto_lower_bound -u <lowerbound> < input > output
\end{lstlisting}
An example on such an input file is:
\begin{lstlisting}
0.123
3.432
2.341
1.323
7.538
4.122
0.242
0.654
5.633
\end{lstlisting}

\subsection*{Output}
The output of the \texttt{gto\char`_lower\char`_bound} program is a set of numbers truncated at the a defined lower bound.\\
Using the input above, an output example for this is the following:
\begin{lstlisting}
Using lower bound: 2
2.000000
3.432000
2.341000
2.000000
7.538000
4.122000
2.000000
2.000000
5.633000
\end{lstlisting}
\section{Program gto\char`_brute\char`_force\char`_string}
The \texttt{gto\char`_brute\char`_force\char`_string} generates all combinations, line by line, for an inputted alphabet and specific size.\\
For help type:
\begin{lstlisting}
./gto_brute_force_string -h
\end{lstlisting}
In the following subsections, we explain the input and output paramters.

\subsection*{Input parameters}

The \texttt{gto\char`_brute\char`_force\char`_string} program needs some paramenters for the computation, namely the alphabet and the key size.\\
The attribution is given according to:
\begin{lstlisting}
Usage: ./gto_brute_force_string [options] [[--] args]
   or: ./gto_brute_force_string [options]

It generates all combinations, line by line, for an inputted alphabet and specific size.

    -h, --help            show this help message and exit

Basic options
    -a, --alphabet=<str>  The input alphabet
    -s, --size=<int>      The combinations size
    > output              Output all the combinations (stdout)

Example: ./gto_brute_force_string-a <alphabet> -s <size> > output
\end{lstlisting}

\subsection*{Output}
The output of the \texttt{gto\char`_brute\char`_force\char`_string} program is a set of all possible word combinations with a defined size, using the input alphabet.\\
Using the input above with the alphabet ''abAB'' with the word size of 3, an output example for this is the following:
\begin{lstlisting}
aaa
aab
aaA
aaB
aba
...
BBb
BBA
BBB
\end{lstlisting}
\section{Program gto\char`_real\char`_to\char`_binary\char`_with\char`_threshold}
The \texttt{gto\char`_real\char`_to\char`_binary\char`_with\char`_threshold} converts a sequence of real numbers into a binary sequence, given a threshold. The numbers below to the threshold will be 0.\\
For help type:
\begin{lstlisting}
./gto_real_to_binary_with_threshold -h
\end{lstlisting}
In the following subsections, we explain the input and output paramters.

\subsection*{Input parameters}

The \texttt{gto\char`_real\char`_to\char`_binary\char`_with\char`_threshold} program needs program needs two streams for the computation, namely the real sequence as input. These numbers should be splitted by lines.\\
The attribution is given according to:
\begin{lstlisting}
Usage: ./gto_real_to_binary_with_threshold [options] [[--] args]
   or: ./gto_real_to_binary_with_threshold [options]

It converts a sequence of real numbers into a binary sequence given a threshold.

    -h, --help                show this help message and exit

Basic options
    -t, --threshold=<dbl>     The threshold in real format
    < input.num               Input numeric file (stdin)
    > output.bin              Output binary file (stdout)

Example: ./gto_real_to_binary_with_threshold -t <threshold> < input.num > output.bin
\end{lstlisting}
An example on such an input file is:
\begin{lstlisting}
12.25
1.2
5.44
5.51
7.97
2.34
8.123
\end{lstlisting}

\subsection*{Output}
The output of the \texttt{gto\char`_real\char`_to\char`_binary\char`_with\char`_threshold} program is a binary sequence.\\
An example, for the input using the threshold of 5.5, is:
\begin{lstlisting}
1
0
0
1
1
0
1
\end{lstlisting}
\section{Program gto\char`_sum}
The \texttt{gto\char`_sum} adds decimal values in file, line by line, splitted by spaces or tabs.\\
For help type:
\begin{lstlisting}
./gto_sum -h
\end{lstlisting}
In the following subsections, we explain the input and output paramters.

\subsection*{Input parameters}

The \texttt{gto\char`_sum} program needs program needs two streams for the computation, namely the input, which is a decimal file.\\
The attribution is given according to:
\begin{lstlisting}
Usage: ./gto_sum [options] [[--] args]
   or: ./gto_sum [options]

It adds decimal values in file, line by line, splitted by spaces or tabs.

    -h, --help        show this help message and exit

Basic options
    < input.num       Input numeric file (stdin)
    > output.num      Output numeric file (stdout)

Optional
    -r, --sumrows     When active, the application adds all the values line by line
    -a, --sumall      When active, the application adds all values

Example: ./gto_sum -a < input.num > output.num
\end{lstlisting}
An example on such an input file is:
\begin{lstlisting}
0.123	5	5
3.432
2.341   3   2
1.323
7.538	5
4.122
0.242 
0.654
5.633	10
\end{lstlisting}

\subsection*{Output}
The output of the \texttt{gto\char`_sum} program is a sum of the elements in the input file.\\
Executing the application with the provided input and with the flag to add only the elements in each row (-r), the output of this execution is:
\begin{lstlisting}
10.123000
3.432000
7.341000
1.323000
12.538000
4.122000
0.242000
0.654000
15.633000
\end{lstlisting}
\section{Program gto\char`_filter}
The \texttt{gto\char`_filter} filters numerical sequences.\\
For help type:
\begin{lstlisting}
./gto_filter -h
\end{lstlisting}
In the following subsections, we explain the input and output paramters.

\subsection*{Input parameters}

The \texttt{gto\char`_filter} program needs program needs two streams for the computation, namely the input, which is a decimal file.\\
The attribution is given according to:
\begin{lstlisting}
Usage: ./gto_filter [options] [[--] args]
   or: ./gto_filter [options]

It filters numerical sequences.

    -h, --help                show this help message and exit

Basic options
    < input.num               Input numeric file (stdin)
    > output.num              Output numeric file (stdout)

Optional
    -w, --windowsize=<int>    Window size (defaut 0)
    -d, --drop=<int>          Discard elements (default 0.0)
    -t, --windowtype=<int>    Window type (0=Hamm, 1=Hann, 2=Black, 3=rec) (default 0 (Hamm))
    -c, --onecolumn           Read from one column
    -p, --printone            Print one column
    -r, --reverse             Reverse mode

Example: ./gto_filter -w <windowsize> -d <drop> -t <windowtype> -c -p -r < input.num > output.num
\end{lstlisting}
An example on such an input file is:
\begin{lstlisting}
TO DO
\end{lstlisting}

\subsection*{Output}
The output of the \texttt{gto\char`_filter} program is ...\\
An example, for the input, is:
\begin{lstlisting}
TO DO
\end{lstlisting}
\section{Program gto\char`_word\char`_search}
The \texttt{gto\char`_word\char`_search} search for a word in a file. It is case sensitive.\\
For help type:
\begin{lstlisting}
./gto_word_search -h
\end{lstlisting}
In the following subsections, we explain the input and output paramters.

\subsection*{Input parameters}

The \texttt{gto\char`_word\char`_search} program needs program needs two streams for the computation, namely the input and output standard. The input stream is a text file.\\
The attribution is given according to:
\begin{lstlisting}
Usage: ./gto_word_search [options] [[--] args]
   or: ./gto_word_search [options]

Searching for a word in a text file. It is case sensitive.

    -h, --help        show this help message and exit

Basic options
    -w, --word=<str>  Word to search in the file
    < input.txt       Input text file (stdin)
    > output.txt      Output text file (stdout)

Example: ./gto_word_search-w <word> < input.txt > output.txt
\end{lstlisting}
An example on such an input file is:
\begin{lstlisting}
No guts, no story. Chris Brady
My life is my message. Mahatma Gandhi
Screw it, let’s do it. Richard Branson
Boldness be my friend. William Shakespeare
Keep going. Be all in. Bryan Hutchinson
My life is my argument. Albert Schweitzer
Fight till the last gasp. William Shakespeare
Leave no stone unturned. Euripides
\end{lstlisting}

\subsection*{Output}
The output of the \texttt{gto\char`_word\char`_search} program is a text file with the matching paragraphs and the location of the word found.\\
Using the input above with the word ''Shakespeare'', an output example for this is the following:
\begin{lstlisting}
Found match in range [ 1536 : 2048 ]
Boldness be my friend. William Shakespeare

Found match in range [ 3072 : 3584 ]
Fight till the last gasp. William Shakespeare
\end{lstlisting}
\section{Program gto\char`_permute\char`_by\char`_blocks}
The \texttt{gto\char`_permute\char`_by\char`_blocks} permutates by block sequence, FASTA and Multi-FASTA files.\\
For help type:
\begin{lstlisting}
./gto_ -h
\end{lstlisting}
In the following subsections, we explain the input and output paramters.

\subsection*{Input parameters}

The \texttt{gto\char`_permute\char`_by\char`_blocks} program needs two streams for the computation, namely the input and output standard. The input stream is a sequence, FASTA or Multi-FASTA file.\\
The attribution is given according to:
\begin{lstlisting}
Usage: ./gto_permute_by_blocks [options] [[--] args]
   or: ./gto_permute_by_blocks [options]

It permutates by block sequence, FASTA and Multi-FASTA files.

    -h, --help            show this help message and exit

Basic options
    -b, --numbases=<int>  The number of bases in each block
    -s, --seed=<int>      Starting point to the random generator
    < input               Input sequence, FASTA or Multi-FASTA file format (stdin)
    > output              Output sequence, FASTA or Multi-FASTA file format (stdout)

Example: ./gto_permute_by_blocks -b <numbases> -s <seed> < input.fasta > output.fasta
\end{lstlisting}
An example of such an input file is:
\begin{lstlisting}
>AB000264 |acc=AB000264|descr=Homo sapiens mRNA 
ACAAGACGGCCTCCTGCTGCTGCTGCTCTCCGGGGCCACGGCCCTGGAGGGTCCACCGCTGCCCTGCTGCCATTGTCCCC
GGCCCCACCTAAGGAAAAGCAGCCTCCTGACTTTCCTCGCTTGGGCCGAGACAGCGAGCATATGCAGGAAGCGGCAGGAA
GTGGTTTGAGTGGACCTCCGGGCCCCTCATAGGAGAGGAAGCTCGGGAGGTGGCCAGGCGGCAGGAAGCAGGCCAGTGCC
GCGAATCCGCGCGCCGGGACAGAATCTCCTGCAAAGCCCTGCAGGAACTTCTTCTGGAAGACCTTCTCCACCCCCCCAGC
TAAAACCTCACCCATGAATGCTCGCAACACGCAAGTTTAATTCGCAAGTTAGACCTGAACGGGAGGTGGCCACGCAAGTT
\end{lstlisting}

\subsection*{Output}
The output of the \texttt{gto\char`_permute\char`_by\char`_blocks} program is a sequence, FASTA or Multi-FASTA file permuted following some parameters.\\
Using the input above with the base number as 80, an output example for this is the following:
\begin{lstlisting}
GGCCCCACCTAAGGAAAAGCAGCCTCCTGACTTTCCTCGCTTGGGCCGAGACAGCGAGCATATGCAGGAAGCGGCAGGAA
GTGGTTTGAGTGGACCTCCGGGCCCCTCATAGGAGAGGAAGCTCGGGAGGTGGCCAGGCGGCAGGAAGCAGGCCAGTGCC
GCGAATCCGCGCGCCGGGACAGAATCTCCTGCAAAGCCCTGCAGGAACTTCTTCTGGAAGACCTTCTCCACCCCCCCAGC
ACAAGACGGCCTCCTGCTGCTGCTGCTCTCCGGGGCCACGGCCCTGGAGGGTCCACCGCTGCCCTGCTGCCATTGTCCCC
TAAAACCTCACCCATGAATGCTCGCAACACGCAAGTTTAATTCGCAAGTTAGACCTGAACGGGAGGTGGCCACGCAAGTT
\end{lstlisting}
\section{Program gto\char`_info}
The \texttt{gto\char`_info} gives the basic properties of the file, namely size, cardinality, distribution percentage of the symbols, among others.\\
For help type:
\begin{lstlisting}
./gto_info -h
\end{lstlisting}
In the following subsections, we explain the input and output paramters.

\subsection*{Input parameters}

The \texttt{gto\char`_info} program needs two streams for the computation,
namely the input and output standard. The input stream is a file withou any specific format.\\
The attribution is given according to:
\begin{lstlisting}
Usage: ./gto_info [options] [[--] args]
   or: ./gto_info [options]

It gives the basic properties of the file, namely size, cardinality, distribution 
percentage of the symbols, among others.

    -h, --help    show this help message and exit

Basic options
    < input       Input file (stdin)
    > output      Output read information (stdout)

Optional
    -a, --ascii   When active, the application shows the ASCII codes

Example: ./gto_info < input > output

Output example :
Number of symbols  : value
Alphabet size      : value
Alphabet           : value
Symbol distribution:
<Symbol/Code ASCII>  <Symbol count>  <Distribution percentage>
\end{lstlisting}
An example on such an input file is:
\begin{lstlisting}
>AB000264 |acc=AB000264|descr=Homo sapiens mRNA 
ACAAGACGGCCTCCTGCTGCTGCTGCTCTCCGGGGCCACGGCCCTGGAGGGTCCACCGCTGCCCTGCTGCCATTGTCCCC
GGCCCCACCTAAGGAAAAGCAGCCTCCTGACTTTCCTCGCTTGGGCCGAGACAGCGAGCATATGCAGGAAGCGGCAGGAA
GTGGTTTGAGTGGACCTCCGGGCCCCTCATAGGAGAGGAAGCTCGGGAGGTGGCCAGGCGGCAGGAAGCAGGCCAGTGCC
GCGAATCCGCGCGCCGGGACAGAATCTCCTGCAAAGCCCTGCAGGAACTTCTTCTGGAAGACCTTCTCCACCCCCCCAGC
TAAAACCTCACCCATGAATGCTCGCAACACGCAAGTTTAATTCGCAAGTTAGACCTGAACGGGAGGTGGCCACGCAAGTT
\end{lstlisting}

\subsection*{Output}
The output of the \texttt{gto\char`_info} program is a set of information related to the file read. \\
Using the input above, an output example for this is the following:
\begin{lstlisting}
Number of symbols  : 453
Alphabet size      : 28
Alphabet           :|srponmiedcaTRNHGCBA>=6420 \n
Symbol distribution:
|  : 	2		0.4415011
s  : 	3		0.66225166
r  : 	1		0.22075055
p  : 	1		0.22075055
o  : 	2		0.4415011
n  : 	1		0.22075055
m  : 	2		0.4415011
i  : 	1		0.22075055
e  : 	2		0.4415011
d  : 	1		0.22075055
c  : 	3		0.66225166
a  : 	2		0.4415011
T  : 	66		14.569536
R  : 	1		0.22075055
N  : 	1		0.22075055
H  : 	1		0.22075055
G  : 	117		25.827815
C  : 	131		28.918322
B  : 	2		0.4415011
A  : 	89		19.646799
>  : 	1		0.22075055
=  : 	2		0.4415011
6  : 	2		0.4415011
4  : 	2		0.4415011
2  : 	2		0.4415011
0  : 	6		1.3245033
   : 	4		0.88300221
\n : 	5		1.1037528
\end{lstlisting}

\section{Program gto\char`_segment}
The \texttt{gto\char`_segment} segments a filtered sequence.\\
For help type:
\begin{lstlisting}
./gto_segment -h
\end{lstlisting}
In the following subsections, we explain the input and output paramters.

\subsection*{Input parameters}

The \texttt{gto\char`_segment} program needs two streams for the computation, namely the input and output standard. The input stream is a numeric file.\\
The attribution is given according to:
\begin{lstlisting}
Usage: ./gto_segment [options] [[--] args]
   or: ./gto_segment [options]

It segments a filtered sequence.

    -h, --help                show this help message and exit

Basic options
    -t, --threshold=<dbl>     The segment threshold
    < input.num               Input numeric file (stdin)
    > output                  Output the segment file (stdout)

Example: ./gto_segment -t <threshold> < input.num > output
\end{lstlisting}
An example on such an input file is:
\begin{lstlisting}
1   1.77
5   2.18
10  2.32
15  3.15
20  2.52
25  4.43
30  1.23
\end{lstlisting}

\subsection*{Output}
The output of the \texttt{gto\char`_segment} program is the interval of values ​​below the threshold.\\
An example, for the input using a thresould of 3, is:
\begin{lstlisting}
0:10
\end{lstlisting}
