\section{Program gto\char`_filter}
The \texttt{gto\char`_filter} filters numerical sequences using a low-pass filter.\\
For help type:
\begin{lstlisting}
./gto_filter -h
\end{lstlisting}
In the following subsections, we explain the input and output paramters.

\subsection*{Input parameters}

The \texttt{gto\char`_filter} program needs two streams for the computation, namely the input and output standard. The input stream is a numeric file.\\
The attribution is given according to:
\begin{lstlisting}
Usage: ./gto_filter [options] [[--] args]
   or: ./gto_filter [options]

It filters numerical sequences using a low-pass filter.

    -h, --help                show this help message and exit

Basic options
    < input.num               Input numeric file (stdin)
    > output.num              Output numeric file (stdout)

Optional
    -w, --windowsize=<int>    Window size (defaut 0)
    -d, --drop=<int>          Discard elements (default 0.0)
    -t, --windowtype=<int>    Window type (0=Hamm, 1=Hann, 2=Black, 3=rec) (default 0 (Hamm))
    -c, --onecolumn           Read from one column
    -p, --printone            Print one column
    -r, --reverse             Reverse mode

Example: ./gto_filter -w <windowsize> -d <drop> -t <windowtype> -c -p -r < input.num > output.num
\end{lstlisting}
An example of such an input file is:
\begin{lstlisting}
1   1.77
5   2.18
10  2.32
15  3.15
20  2.52
25  4.43
30  1.23
\end{lstlisting}

\subsection*{Output}
The output of the \texttt{gto\char`_filter} program is a numeric file, identical of the input.\\
Using the input above with the window size of 3, an output example for this is the following:
\begin{lstlisting}
Got 7 entries from file
1   2.085
5   2.256
10  2.507
15  2.757
20  2.905
25  2.860
30  2.674
\end{lstlisting}