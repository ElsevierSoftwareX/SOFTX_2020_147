\section{Program goose-newlineonnewx}
The \texttt{goose-newlineonnewx} splits different rows with a new empty row.\\
For help type:
\begin{lstlisting}
./goose-newlineonnewx -h
\end{lstlisting}
In the following subsections, we explain the input and output paramters.

\subsection*{Input parameters}

The \texttt{goose-newlineonnewx} program needs two streams for the computation, namely the input and output standard. The input stream is a matrix file format with 3 columns.\\
The attribution is given according to:
\begin{lstlisting}
Usage: ./goose-newlineonnewx [options] [[--] args]
   or: ./goose-newlineonnewx [options]

It splits different rows with a new empty row.

    -h, --help    show this help message and exit

Basic options
    < input       Input file with 3 column matrix format (stdin)
    > output      Output file with 3 column matrix format (stdout)

Example: ./goose-newlineonnewx < input > output
\end{lstlisting}
An example on such an input file is:
\begin{lstlisting}
1	2	2
1	2	2
4	4	1
10	12	2
15	15	1
45	47	3
45	47	3
45	47	3
45	47	3
55	55	1
\end{lstlisting}

\subsection*{Output}
The output of the \texttt{goose-newlineonnewx} program is a 3 column matrix, with an empty line between different rows.\\
An example, for the input, is:
\begin{lstlisting}
1.000000	2.000000	2.000000
1.000000	2.000000	2.000000

4.000000	4.000000	1.000000

10.000000	12.000000	2.000000

15.000000	15.000000	1.000000

45.000000	47.000000	3.000000
45.000000	47.000000	3.000000
45.000000	47.000000	3.000000
45.000000	47.000000	3.000000

55.000000	55.000000	1.000000
\end{lstlisting}
