\section{Program gto\char`_fastq\char`_maximum\char`_read\char`_size}
The \texttt{gto\char`_fastq\char`_maximum\char`_read\char`_size} filters the FASTQ reads with the length higher than the value defined.\\
For help type:
\begin{lstlisting}
./gto_fastq_maximum_read_size -h
\end{lstlisting}
In the following subsections, we explain the input and output paramters.

\subsection*{Input parameters}

The \texttt{gto\char`_fastq\char`_maximum\char`_read\char`_size} program needs two streams for the computation,
namely the input and output standard. The input stream is a FASTQ file.\\
The attribution is given according to:
\begin{lstlisting}
Usage: ./gto_fastq_maximum_read_size [options] [[--] args]
   or: ./gto_fastq_maximum_read_size [options]

It filters the FASTQ reads with the length higher than the value defined. 
If present, it will erase the second header (after +).

    -h, --help            show this help message and exit

Basic options
    -s, --size=<int>      The maximum read length
    < input.fastq         Input FASTQ file format (stdin)
    > output              Output read information (stdout)

Example: ./gto_fastq_maximum_read_size < input.fastq > output

Output example :
<FASTQ non-filtered reads>
Total reads    : value
Filtered reads : value
\end{lstlisting}
An example on such an input file is:
\begin{lstlisting}
@SRR001666.1 071112_SLXA-EAS1_s_7:5:1:817:345 length=60
GGGTGATGGCCGCTGCCGATGGCGTCAAATCCCACCAAGTTACCCTTAACAACTTAAGGG
+SRR001666.1 071112_SLXA-EAS1_s_7:5:1:817:345 length=60
IIIIIIIIIIIIIIIIIIIIIIIIIIIIII9IG9ICIIIIIIIIIIIIIIIIIIIIDIII
@SRR001666.2 071112_SLXA-EAS1_s_7:5:1:801:338 length=72
GTTCAGGGATACGACGTTTGTATTTTAAGAATCTGAAGCAGAAGTCGATGATAATACGCGTCGTTTTATCAT
+SRR001666.2 071112_SLXA-EAS1_s_7:5:1:801:338 length=72
IIIIIIIIIIIIIIIIIIIIIIIIIIIIIIII6IBIIIIIIIIIIIIIIIIIIIIIIIGII>IIIII-I)8I
\end{lstlisting}

\subsection*{Output}
The output of the \texttt{gto\char`_fastq\char`_maximum\char`_read\char`_size} program is a set of all the filtered FASTQ reads, followed by the execution report.\\
Using the size value as 60, an example for this input, is: 
\begin{lstlisting}
@SRR001666.1 071112_SLXA-EAS1_s_7:5:1:817:345 length=60
GGGTGATGGCCGCTGCCGATGGCGTCAAATCCCACCAAGTTACCCTTAACAACTTAAGGG
+
IIIIIIIIIIIIIIIIIIIIIIIIIIIIII9IG9ICIIIIIIIIIIIIIIIIIIIIDIII
Total reads    : 2
Filtered reads : 1
\end{lstlisting}
