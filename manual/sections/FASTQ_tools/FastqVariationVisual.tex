\section{Program gto\char`_fastq\char`_variation\char`_visual}
The \texttt{gto\char`_fastq\char`_variation\char`_visual} is TO DO\\
For help type:
\begin{lstlisting}
./gto_fastq_variation_visual -h
\end{lstlisting}
In the following subsections, we explain the input and output paramters.

\subsection*{Input parameters}

The \texttt{gto\char`_fastq\char`_variation\char`_visual} program needs program needs a FASTQ file to compute.\\
The attribution is given according to:
\begin{lstlisting}
Usage: ./gto_fastq_variation_visual <OPTIONS>... [FILE]:<...>
./gto_fastq_variation_visual: visualize relative singularity regions.
                                                     
  -v                       verbose mode,             
  -a                       about CHESTER,            
  -e <value>               enlarge painted regions,  
                                                     
  [tFile1]:<tFile2>:<...>  target file(s).           
                                                     
Report bugs to <{pratas,raquelsilva,ap,pjf}@ua.pt>. 
\end{lstlisting}
An example of such an input file is:
\begin{lstlisting}
TO DO
\end{lstlisting}

\subsection*{Output}
The output of the \texttt{gto\char`_fastq\char`_variation\char`_visual} program is a \\
TO DO
\begin{lstlisting}
TO DO
\end{lstlisting}