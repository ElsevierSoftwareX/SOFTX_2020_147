\section{Program gto\char`_fastq\char`_falcon}
The \texttt{gto\char`_fastq\char`_falcon} is an ultra-fast method to infer metagenomic composition of sequenced reads. gto\char`_fastq\char`_falcon measures similarity between any FASTQ file (or FASTA), independently from the size, against any multi-FASTA database, such as the entire set of complete genomes from the NCBI. gto\char`_fastq\char`_falcon supports single reads, paired-end reads, and compositions of both. It has been tested in many plataforms, such as Illumina MySeq, HiSeq, Novaseq, IonTorrent.\\
gto\char`_fastq\char`_falcon is efficient to detect the presence and authenticate a given species in the FASTQ reads. The core of the method is based on relative data compression. gto\char`_fastq\char`_falcon uses variable multi-threading, without multiplying the memory for each thread, being able to run efficiently in a common laptop.\\
For help type:
\begin{lstlisting}
./gto\char`_fastq\char`_falcon -h
\end{lstlisting}
In the following subsections, we explain the input and output paramters.

\subsection*{Input parameters}

The \texttt{gto\char`_fastq\char`_falcon} program needs program needs a FASTQ file to compute.\\
The attribution is given according to:
\begin{lstlisting}
NAME                                                                     
      gto\char`_fastq\char`_falcon v3.1: a tool to infer metagenomic composition.            
                                                                         
SYNOPSIS                                                                 
      gto\char`_fastq\char`_falcon [OPTION]... [FILE1]:[FILE2]:... [FILE]                      
                                                                         
SAMPLE                                                                   
      gto\char`_fastq\char`_falcon -v -F -l 47 -Z -y pro.com reads1.fq:reads2.fq DB.fa         
                                                                         
DESCRIPTION                                                              
      It infers metagenomic sample composition of sequenced reads.       
      The core of the method uses a cooperation between multiple         
      context and tolerant context models with several depths.           
      The reference sequences must be in a multi-FASTA format.           
      The sequenced reads must be trimmed and in FASTQ format.           
                                                                         
      Non-mandatory arguments:                                           
                                                                         
      -h                   give this help,                               
      -F                   force mode (overwrites top file),             
      -V                   display version number,                       
      -v                   verbose mode (more information),              
      -Z                   database local similarity,                    
      -s                   show compression levels,                      
                                                                         
      -l <level>           compression level [1;47],                    
      -p <sample>          subsampling (default: 1),                    
      -t <top>             top of similarity (default: 20),              
      -n <nThreads>        number of threads (default: 2),              
                                                                         
      -x <FILE>            similarity top filename,                      
      -y <FILE>            profile filename (-Z must be on).             
                                                                         
      Mandatory arguments:                                               
                                                                         
      [FILE1]:[FILE2]:...  metagenomic filename (FASTQ),                 
                           Use ":" for splitting files.                
                                                                         
      [FILE]               database filename (Multi-FASTA).              
                                                                         
COPYRIGHT                                                                
      Copyright (C) 2014-2019, IEETA, University of Aveiro.              
      This is a Free software, under GPLv3. You may redistribute         
      copies of it under the terms of the GNU - General Public           
      License v3 <http://www.gnu.org/licenses/gpl.html>.    
\end{lstlisting}
An example of such an input file is:
\begin{lstlisting}
TO DO
\end{lstlisting}

\subsection*{Output}
The output of the \texttt{gto\char`_fastq\char`_falcon} program is a FASTQ file\\
TO DO
\begin{lstlisting}
TO DO
\end{lstlisting}