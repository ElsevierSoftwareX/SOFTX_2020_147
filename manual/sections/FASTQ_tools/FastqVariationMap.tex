\section{Program gto\char`_fastq\char`_variation\char`_map}
The \texttt{gto\char`_fastq\char`_variation\char`_map} is TO DO\\
For help type:
\begin{lstlisting}
./gto_fastq_variation_map -h
\end{lstlisting}
In the following subsections, we explain the input and output paramters.

\subsection*{Input parameters}

The \texttt{gto\char`_fastq\char`_variation\char`_map} program needs program needs a FASTQ file to compute.\\
The attribution is given according to:
\begin{lstlisting}
Usage: ./gto_fastq_variation_map <OPTIONS>... [FILE]:<...> [FILE]:<...>
./gto_fastq_variation_map: a tool to map relative singularity regions  
The (probabilistic) Bloom filter is automatically set.   
                                                         
  -v                       verbose mode,                 
  -a                       about CHESTER,                
  -s <value>               bloom size,                   
  -i                       use inversions,               
  -p                       show positions/words,         
  -k <value>               k-mer size (up to 30),        
                                                         
  [rFile1]:<rFile2>:<...>  reference file(s),            
  [tFile1]:<tFile2>:<...>  target file(s).               
                                                         
The reference files may be FASTA, FASTQ or DNA SEQ,      
while the target files may be FASTA OR DNA SEQ.          
Report bugs to <{pratas,raquelsilva,ap,pjf}@ua.pt>.
\end{lstlisting}
An example of such an input file is:
\begin{lstlisting}
TO DO
\end{lstlisting}

\subsection*{Output}
The output of the \texttt{gto\char`_fastq\char`_variation\char`_map} program is a \\
TO DO
\begin{lstlisting}
TO DO
\end{lstlisting}