\section{Program gto\char`_fastq\char`_minimum\char`_local\char`_quality\char`_score\char`_forward}
The \texttt{gto\char`_fastq\char`_minimum\char`_local\char`_quality\char`_score\char`_forward} filters the reads considering the quality score average of a defined window size of bases.\\
For help type:
\begin{lstlisting}
./gto_fastq_minimum_local_quality_score_forward -h
\end{lstlisting}
In the following subsections, we explain the input and output paramters.

\subsection*{Input parameters}

The \texttt{gto\char`_fastq\char`_minimum\char`_local\char`_quality\char`_score\char`_forward} program needs program needs two streams for the computation, namely the input and output standard. The input stream is a FASTQ file.\\
The attribution is given according to:
\begin{lstlisting}
Usage: ./gto_fastq_minimum_local_quality_score_forward [options] [[--] args]
   or: ./gto_fastq_minimum_local_quality_score_forward [options]

It filters the reads considering the quality score average of a defined window size 
of bases.

    -h, --help                show this help message and exit

Basic options
    -k, --windowsize=<int>    The window size of bases (default 5)
    -w, --minavg=<int>        The minimum average of quality score (default 25)
    -m, --minqs=<int>         The minimum value of the quality score (default 33)
    < input.fastq             Input FASTQ file format (stdin)
    > output.fastq            Output FASTQ file format (stdout)

Example: ./gto_fastq_minimum_local_quality_score_forward -k <windowsize> -w <minavg> 
-m <minqs> < input.fastq > output.fastq

Console output example:
Minimum QS       : value
<FASTQ output>
Total reads      : value
Trimmed reads    : value
\end{lstlisting}
An example on such an input file is:
\begin{lstlisting}
@SRR001666.1 071112_SLXA-EAS1_s_7:5:1:817:345 length=72
GGGTGATGGCCGCTGCCGATGGCGTCAAATCCCACCAAGTTACCCTTAACAACTTAAGGGTTTTCAAATAGA
+SRR001666.1 071112_SLXA-EAS1_s_7:5:1:817:345 length=72
IIIIIIIIIIIIIIIIIIIIIIIIIIIIII9IG9ICIIIIIIIIIIIIIIIIIIIIDIIIIIII>IIIIII/
@SRR001666.2 071112_SLXA-EAS1_s_7:5:1:801:338 length=72
GTTCAGGGATACGACGTTTGTATTTTAAGAATCTGAAGCAGAAGTCGATGATAATACGCGTCGTTTTATCAT
+SRR001666.2 071112_SLXA-EAS1_s_7:5:1:801:338 length=72
IIIIIIIIIIIIIIIIIIIIIIIIIIIIIIII6IBIIIIIIIIIIIIIIIIIIIIIIIGII>IIIII-I)8I
\end{lstlisting}

\subsection*{Output}
The output of the \texttt{gto\char`_fastq\char`_minimum\char`_local\char`_quality\char`_score\char`_forward} program is a FASTQ file with the reads filtered following a quality score average of a defined window of bases.
The execution report only appears in the console.\\
Using the input above with the default values, an output example for this is the following:
\begin{lstlisting}
Minimum QS     : 33
@SRR001666.1 071112_SLXA-EAS1_s_7:5:1:817:345 length=72
GGGTGATGGCCGCTGCCGATGGCGTCAAATCCCACCAAGTTACCCTTAACAACTTAAGGGTTTTCAAATAGA
+
IIIIIIIIIIIIIIIIIIIIIIIIIIIIII9IG9ICIIIIIIIIIIIIIIIIIIIIDIIIIIII>IIIIII/
@SRR001666.2 071112_SLXA-EAS1_s_7:5:1:801:338 length=72
GTTCAGGGATACGACGTTTGTATTTTAAGAATCTGAAGCAGAAGTCGATGATAATACGCGTCGTTT
+
IIIIIIIIIIIIIIIIIIIIIIIIIIIIIIII6IBIIIIIIIIIIIIIIIIIIIIIIIGII>IIII
Total reads    : 2
Trimmed reads  : 1
\end{lstlisting}