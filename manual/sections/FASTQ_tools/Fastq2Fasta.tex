\section{Program GTO-Fastq2Fasta}
The \texttt{GTO-Fastq2Fasta} converts a FASTQ file format to a pseudo FASTA file. However, it does not align the sequence. Also, it extracts the sequence and adds a pseudo header.\\
For help type:
\begin{lstlisting}
./GTO-Fastq2Fasta -h
\end{lstlisting}
In the following subsections, we explain the input and output paramters.

\subsection*{Input parameters}

The \texttt{GTO-Fastq2Fasta} program needs two streams for the computation,
namely the input and output standard. The input stream is a FASTQ file.\\
The attribution is given according to:
\begin{lstlisting}
Usage: ./GTO-Fastq2Fasta [options] [[--] args]
   or: ./GTO-Fastq2Fasta [options]

It converts a FASTQ file format to a pseudo FASTA file.
It does NOT align the sequence.
It extracts the sequence and adds a pseudo header.

    -h, --help            show this help message and exit

Basic options
    < input.fastq         Input FASTQ file format (stdin)
    > output.fasta        Output FASTA file format (stdout)

Example: ./GTO-Fastq2Fasta < input.fastq > output.fasta
\end{lstlisting}
An example on such an input file is:
\begin{lstlisting}
@SRR001666.1 071112_SLXA-EAS1_s_7:5:1:817:345 length=72
GGGTGATGGCCGCTGCCGATGGCGTCAAATCCCACCAAGTTACCCTTAACAACTTAAGGGTTTTCAAATAGA
+SRR001666.1 071112_SLXA-EAS1_s_7:5:1:817:345 length=72
IIIIIIIIIIIIIIIIIIIIIIIIIIIIII9IG9ICIIIIIIIIIIIIIIIIIIIIDIIIIIII>IIIIII/
@SRR001666.2 071112_SLXA-EAS1_s_7:5:1:801:338 length=72
GTTCAGGGATACGACGTTTGTATTTTAAGAATCTGAAGCAGAAGTCGATGATAATACGCGTCGTTTTATCAT
+SRR001666.2 071112_SLXA-EAS1_s_7:5:1:801:338 length=72
IIIIIIIIIIIIIIIIIIIIIIIIIIIIIIII6IBIIIIIIIIIIIIIIIIIIIIIIIGII>IIIII-I)8I
\end{lstlisting}

\subsection*{Output}
The output of the \texttt{GTO-Fastq2Fasta} program a FASTA file.\\
An example, for the input, is:
\begin{lstlisting}
GGGTGATGGCCGCTGCCGATGGCGTCAAATCCCACCAAGTTACCCTTAACAACTTAAGGGTTTTCAAATAGA
GTTCAGGGATACGACGTTTGTATTTTAAGAATCTGAAGCAGAAGTCGATGATAATACGCGTCGTTTTATCAT
\end{lstlisting}
