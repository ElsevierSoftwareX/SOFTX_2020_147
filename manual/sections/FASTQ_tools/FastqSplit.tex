\section{Program gto\char`_fastq\char`_split}
The \texttt{gto\char`_fastq\char`_split} splits Paired End files according to the direction of the strand ('/1' or '/2'). It writes by default singleton reads as forward stands. \\
For help type:
\begin{lstlisting}
./gto_fastq_split -h
\end{lstlisting}
In the following subsections, we explain the input and output paramters.

\subsection*{Input parameters}

The \texttt{gto\char`_fastq\char`_split} program needs a stream for the computation,
namely the input standard. The input stream is a FASTQ file.\\
The attribution is given according to:
\begin{lstlisting}
Usage: ./gto_fastq_split [options] [[--] args]
   or: ./gto_fastq_split [options]

It writes by default singleton reads as forward stands.

    -h, --help            show this help message and exit

Basic options
    -f, --forward=<str>   Output forward file
    -r, --reverse=<str>   Output reverse file
    < input.fastq         Input FASTQ file format (stdin)
    > output         	  Output read information (stdout)

Example: ./gto_fastq_split -t <output_forward.fastq> -r <output_reverse.fastq> < input.fastq > output

Output example :
Total reads      : value
Singleton reads  : value
Forward reads    : value
Reverse reads    : value
\end{lstlisting}
An example on such an input file is:
\begin{lstlisting}
@SRR001666.1 071112_SLXA-EAS1_s_7:5:1:817:345 length=72 1
GNNTGATGGCCGCTGCCGATGGCGNANAATCCCACCAANATACCCTTAACAACTTAAGGGTTNTCAAATAGA
+
IIIIIIIIIIIIIIIIIIIIIIIIIIIIII9IG9ICIIIIIIIIIIIIIIIIIIIIDIIIIIII>IIIIII/
@SRR001666.2 071112_SLXA-EAS1_s_7:5:1:801:338 length=72 2
NTTCAGGGATACGACGNTTGTATTTTAAGAATCTGNAGCAGAAGTCGATGATAATACGCGNCGTTTTATCAN
+
IIIIIIIIIIIIIIIIIIIIIIIIIIIIIIII6IBIIIIIIIIIIIIIIIIIIIIIIIGII>IIIII-I)8I
\end{lstlisting}

\subsection*{Output}
The output of the \texttt{gto\char`_fastq\char`_split} program is a set of informations related with the file readed.\\
An example, for the input, is:
\begin{lstlisting}
Total reads     : 2
Singleton reads : 0
Forward reads   : 65536
Reverse reads   : 1
\end{lstlisting}
Also, this program generates two FASTQ files, with the reverse and forward reads.\\
An example of the forward reads, for the input, is: 
\begin{lstlisting}
@SRR001666.1 071112_SLXA-EAS1_s_7:5:1:817:345 length=72 1
GNNTGATGGCCGCTGCCGATGGCGNANAATCCCACCAANATACCCTTAACAACTTAAGGGTTNTCAAATAGA
+
IIIIIIIIIIIIIIIIIIIIIIIIIIIIII9IG9ICIIIIIIIIIIIIIIIIIIIIDIIIIIII>IIIIII/
\end{lstlisting}
