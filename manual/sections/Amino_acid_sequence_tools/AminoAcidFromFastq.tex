\section{Program gto\char`_amino\char`_acid\char`_from\char`_fastq}
The \texttt{gto\char`_amino\char`_acid\char`_from\char`_fastq} converts DNA sequences in the FASTQ file format to an amino acid sequence.\\
For help type:
\begin{lstlisting}
./gto_amino_acid_from_fastq -h
\end{lstlisting}
In the following subsections, we explain the input and output paramters.

\subsection*{Input parameters}

The \texttt{gto\char`_amino\char`_acid\char`_from\char`_fastq} program needs two streams for the computation, namely the input and output standard. The input stream is a FASTQ file.\\
The attribution is given according to:
\begin{lstlisting}
Usage: ./gto_amino_acid_from_fastq [options] [[--] args]
   or: ./gto_amino_acid_from_fastq [options]

It converts DNA sequences in the FASTQ file format to an amino acid sequence.

    -h, --help            Show this help message and exit

Basic options
    < input.fastq         Input FASTQ file format (stdin)
    > output.prot         Output amino acid sequence file (stdout)

Example: ./gto_amino_acid_from_fastq < input.fastq > output.prot
\end{lstlisting}
An example of such an input file is:
\begin{lstlisting}
@SRR001666.1 071112_SLXA-EAS1_s_7:5:1:817:345 length=72
GGGTGATGGCCGCTGCCGATGGCGTCAAATCCCACCAAGTTACCCTTAACAACTTAAGGGTTTTCAAATAGA
+SRR001666.1 071112_SLXA-EAS1_s_7:5:1:817:345 length=72
IIIIIIIIIIIIIIIIIIIIIIIIIIIIII9IG9ICIIIIIIIIIIIIIIIIIIIIDIIIIIII>IIIIII/
@SRR001666.2 071112_SLXA-EAS1_s_7:5:1:801:338 length=72
GTTCAGGGATACGACGTTTGTATTTTAAGAATCTGAAGCAGAAGTCGATGATAATACGCGTCGTTTTATCAT
+SRR001666.2 071112_SLXA-EAS1_s_7:5:1:801:338 length=72
IIIIIIIIIIIIIIIIIIIIIIIIIIIIIIII6IBIIIIIIIIIIIIIIIIIIIIIIIGII>IIIII-I)8I
\end{lstlisting}

\subsection*{Output}

The output of the \texttt{gto\char`_amino\char`_acid\char`_from\char`_fastq} program is an amino acid sequence.\\
Using the input above, an output example for this is the following:
\begin{lstlisting}
TO DO
\end{lstlisting}
