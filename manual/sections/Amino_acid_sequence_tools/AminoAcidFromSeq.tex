\section{Program gto\char`_amino\char`_acid\char`_from\char`_seq}
The \texttt{gto\char`_amino\char`_acid\char`_from\char`_seq} converts DNA sequence to an amino acid sequence.\\
For help type:
\begin{lstlisting}
./gto_amino_acid_from_seq -h
\end{lstlisting}
In the following subsections, we explain the input and output paramters.

\subsection*{Input parameters}

The \texttt{gto\char`_amino\char`_acid\char`_from\char`_seq} program needs two streams for the computation, namely the input and output standard. The input stream is a DNA sequence.\\
The attribution is given according to:
\begin{lstlisting}
Usage: ../../bin/gto_amino_acid_from_seq [options] [[--] args]
   or: ../../bin/gto_amino_acid_from_seq [options]

It converts DNA sequence to an amino acid sequence (translation).


    -h, --help            Show this help message and exit

Basic options
    < input.seq           Input sequence file (stdin)
    > output.prot         Output amino acid sequence file (stdout)

Optional
    -f, --frame=<int>     Translation codon frame (1, 2 or 3)

Example: ../../bin/gto_amino_acid_from_seq < input.seq > output.prot
\end{lstlisting}
An example of such an input file is:
\begin{lstlisting}
ACAAGACGGCCTCCTGCTGCTGCTGCTCTCCGGGGCCACGGCCCTGGAGGGTCCACCGCTGCCCTGCTGCCATTGTCCC
CGGCCCCACCTAAGGAAAAGCAGCCTCCTGACTTTCCTCGCTTGGGCCGAGACAGCGAGCATATGCAGGAAGCGGCAGG
AAGTGGTTTGAGTGGACCTCCGGGCCCCTCATAGGAGAGGAAGCTCGGGAGGTGGCCAGGCGGCAGGAAGCAGGCCAGT
GCCGCGAATCCGCGCGCCGGGACAGAATCTCCTGCAAAGCCCTGCAGGAACTTCTTCTGGAAGACCTTCTCCACCCCCC
CAGCTAAAACCTCACCCATGAATGCTCACGCAAGTTTAATTACAGACCTGAAACAAGATGCCATTGTCCCCCGGCCTCC
TGCTGCTGCTGCTCTCCGGGGCCACGGCCACCGCTGCCCTGCCCCTGGAGGGTGGCCCCACCGGCCGAGACAGCGAGCA
TATGCAGGAAGCGGCAGGAATAAGGAAAAGCAGCCTCCTGACTTTCCTCGCTTGGTGGTTTGAGTGGACCTCCCAGGCC
AGTGCCGGGCCCCTCATAGGAGAGGAAGCTCGGGAGGTGGCCAGGCGGCAGGAAGGCGCACCCCCCCAGCAATCCGCGC
GCCGGGACAGAATGCCCTGCAGGAACTTCTTCTGGAAGACCTTCTCCTCCTGCAAATAAAACCTCACCCATGAATGCTC
ACGCAAGTTTAATTACAGACCTGAA
\end{lstlisting}

\subsection*{Output}

The output of the \texttt{gto\char`_amino\char`_acid\char`_from\char`_seq} program is an amino acid sequence.\\
Using the input above, an output example for this is the following:
\begin{lstlisting}
TRRPPAAAALRGHGPGGSTAALLPLSPAPPKEKQPPDFPRLGRDSEHMQEAAGSGLSGPPGPS-ERKLGRWPGGRKQAS
AANPRAGTESPAKPCRNFFWKTFSTPPAKTSPMNAHASLITDLKQDAIVPRPPAAAALRGHGHRCPAPGGWPHRPRQRA
YAGSGRNKEKQPPDFPRLVV-VDLPGQCRAPHRRGSSGGGQAAGRRTPPAIRAPGQNALQELLLEDLLLLQIKPHP-ML
TQV-LQT-
\end{lstlisting}
