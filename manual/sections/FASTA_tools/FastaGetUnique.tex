\section{Program gto\char`_fasta\char`_get\char`_unique}
The \texttt{gto\char`_fasta\char`_get\char`_unique} changes the headers of FASTA or Multi-FASTA file to simple chrX by order, where X is the number.\\
For help type:
\begin{lstlisting}
./gto_fasta_get_unique -h
\end{lstlisting}
In the following subsections, we explain the input and output paramters.

\subsection*{Input parameters}

The \texttt{gto\char`_fasta\char`_get\char`_unique} program needs two streams for the computation, namely the input and output standard. The input stream is a Multi-FASTA file.\\
The attribution is given according to:
\begin{lstlisting}
Usage: ./gto_fasta_get_unique [options] [[--] args]
   or: ./gto_fasta_get_unique [options]

It extracts unique reads from Multi-FASTA files.

    -h, --help            show this help message and exit

Basic options
    < input.fasta         Input Multi-FASTA file format (stdin)
    > output.fasta        Output FASTA or Multi-FASTA file format (stdout)

Example: ./gto_fasta_get_unique < input.fasta > output.fasta
\end{lstlisting}
An example on such an input file is:
\begin{lstlisting}
todo
\end{lstlisting}

\subsection*{Output}
The output of the \texttt{gto\char`_fasta\char`_get\char`_unique} program is a FASTA or Multi-FASTA file.\\
Using the input above, an output example for this is the following:
\begin{lstlisting}
todo
\end{lstlisting}
