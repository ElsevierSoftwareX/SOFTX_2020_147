\section{Program gto\char`_fasta\char`_info}
The \texttt{gto\char`_fasta\char`_info} shows the readed information of a FASTA or Multi-FASTA file format.\\
For help type:
\begin{lstlisting}
./gto_fasta_info -h
\end{lstlisting}
In the following subsections, we explain the input and output paramters.

\subsection*{Input parameters}

The \texttt{gto\char`_fasta\char`_info} program needs two streams for the computation,
namely the input and output standard. The input stream is a FASTA or Multi-FASTA file.\\
The attribution is given according to:
\begin{lstlisting}
Usage: ./gto_fasta_info [options] [[--] args]
   or: ./gto_fasta_info [options]

It shows read information of a FASTA or Multi-FASTA file format.

    -h, --help            show this help message and exit

Basic options
    < input.fasta         Input FASTA or Multi-FASTA file format (stdin)
    > output              Output read information (stdout)

Example: ./gto_fasta_info < input.fasta > output

Output example :
Number of reads      : value
Number of bases      : value
MIN of bases in read : value
MAX of bases in read : value
AVG of bases in read : value
\end{lstlisting}
An example on such an input file is:
\begin{lstlisting}
>AB000264 |acc=AB000264|descr=Homo sapiens mRNA 
ACAAGACGGCCTCCTGCTGCTGCTGCTCTCCGGGGCCACGGCCCTGGAGGGTCCACCGCTGCCCTGCTGCCATTGTCCCC
GGCCCCACCTAAGGAAAAGCAGCCTCCTGACTTTCCTCGCTTGGGCCGAGACAGCGAGCATATGCAGGAAGCGGCAGGAA
GTGGTTTGAGTGGACCTCCGGGCCCCTCATAGGAGAGGAAGCTCGGGAGGTGGCCAGGCGGCAGGAAGCAGGCCAGTGCC
GCGAATCCGCGCGCCGGGACAGAATCTCCTGCAAAGCCCTGCAGGAACTTCTTCTGGAAGACCTTCTCCACCCCCCCAGC
TAAAACCTCACCCATGAATGCTCACGCAAGTTTAATTACAGACCTGAA
>AB000263 |acc=AB000263|descr=Homo sapiens mRNA 
ACAAGATGCCATTGTCCCCCGGCCTCCTGCTGCTGCTGCTCTCCGGGGCCACGGCCACCGCTGCCCTGCCCCTGGAGGGT
GGCCCCACCGGCCGAGACAGCGAGCATATGCAGGAAGCGGCAGGAATAAGGAAAAGCAGCCTCCTGACTTTCCTCGCTTG
GTGGTTTGAGTGGACCTCCCAGGCCAGTGCCGGGCCCCTCATAGGAGAGGAAGCTCGGGAGGTGGCCAGGCGGCAGGAAG
GCGCACCCCCCCAGCAATCCGCGCGCCGGGACAGAATGCCCTGCAGGAACTTCTTCTGGAAGACCTTCTCCTCCTGCAAA
TAAAACCTCACCCATGAATGCTCACGCAAGTTTAATTACAGACCTGAA
\end{lstlisting}

\subsection*{Output}
The output of the \texttt{gto\char`_fasta\char`_info} program is a set of informations related with the file readed. \\
An example, for the input, is:
\begin{lstlisting}
Number of reads      : 2
Number of bases      : 736
MIN of bases in read : 368
MAX of bases in read : 368
AVG of bases in read : 368.0000
\end{lstlisting}