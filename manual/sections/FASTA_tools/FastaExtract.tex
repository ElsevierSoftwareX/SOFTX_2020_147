\section{Program gto\char`_fasta\char`_extract}
The \texttt{gto\char`_fasta\char`_extract} extracts sequences from a FASTA file, which the range is defined by the user in the parameters.\\
For help type:
\begin{lstlisting}
./gto_fasta_extract -h
\end{lstlisting}
In the following subsections, we explain the input and output paramters.

\subsection*{Input parameters}

The \texttt{gto\char`_fasta\char`_extract} program needs two paramenters, which defines the begin and the end of the extraction, and two streams for the computation, namely the input and output standard. The input stream is a FASTA file.\\
The attribution is given according to:
\begin{lstlisting}
Usage: ./gto_fasta_extract [options] [[--] args]
   or: ./gto_fasta_extract [options]

It extracts sequences from a FASTA file.

    -h, --help            show this help message and exit

Basic options
    -i, --init=<int>      The first position to start the extraction (default 0)
    -e, --end=<int>       The last extract position (default 100)
    < input.fasta         Input FASTA or Multi-FASTA file format (stdin)
    > output.seq          Output sequence file (stdout)

Example: ./gto_fasta_extract -i <init> -e <end> < input.fasta > output.seq
\end{lstlisting}
An example on such an input file is:
\begin{lstlisting}
>AB000264 |acc=AB000264|descr=Homo sapiens mRNA 
ACAAGACGGCCTCCTGCTGCTGCTGCTCTCCGGGGCCACGGCCCTGGAGGGTCCACCGCTGCCCTGCTGCCATTGTCCCC
GGCCCCACCTAAGGAAAAGCAGCCTCCTGACTTTCCTCGCTTGGGCCGAGACAGCGAGCATATGCAGGAAGCGGCAGGAA
GTGGTTTGAGTGGACCTCCGGGCCCCTCATAGGAGAGGAAGCTCGGGAGGTGGCCAGGCGGCAGGAAGCAGGCCAGTGCC
GCGAATCCGCGCGCCGGGACAGAATCTCCTGCAAAGCCCTGCAGGAACTTCTTCTGGAAGACCTTCTCCACCCCCCCAGC
TAAAACCTCACCCATGAATGCTCACGCAAGTTTAATTACAGACCTGAA
\end{lstlisting}

\subsection*{Output}
The output of the \texttt{gto\char`_fasta\char`_extract} program is a group sequence.\\
An example, using the value 0 as extraction starting point and the 50 as the end, for the provided input, is:
\begin{lstlisting}
ACAAGACGGCCTCCTGCTGCTGCTGCTCTCCGGGGCCACGGCCCTGGAGG
\end{lstlisting}