\section{Program gto\char`_genomic\char`_count\char`_bases}
The \texttt{gto\char`_genomic\char`_count\char`_bases} counts the number of bases in sequence, FASTA or FASTQ files.\\
For help type:
\begin{lstlisting}
./gto_genomic_count_bases -h
\end{lstlisting}
In the following subsections, we explain the input and output paramters.

\subsection*{Input parameters}

The \texttt{gto\char`_genomic\char`_count\char`_bases} program needs program needs two streams for the computation, namely the input and output standard. The input stream is a sequence, FASTA or FASTQ file.\\
The attribution is given according to:
\begin{lstlisting}
Usage: ./gto_genomic_count_bases [options] [[--] args]
   or: ./gto_genomic_count_bases [options]

It counts the number of bases in sequence, FASTA or FASTQ files.

    -h, --help    Show this help message and exit

Basic options
    < input       Input sequence, FASTA or FASTQ file format (stdin)
    > output      Output read information (stdout)

Example: ./gto_genomic_count_bases < input.seq > output

Output example :
File type        : value
Number of bases  : value
Number of a/A    : value
Number of c/C    : value
Number of g/G    : value
Number of t/T    : value
Number of n/N    : value
Number of others : value
\end{lstlisting}
An example of such an input file is:
\begin{lstlisting}
TCTTTACTCGCGCGTTGGAGAAATACAATAGTGCGGCTCTGTCTCCTTATGAAGTCAACAATTTCGCTGGGACTTGCGGC
TCTTTACTCGCGCGTTGGAGAAATACAATAGTGCGGCTCTGTCTCCTTATGAAGTCAACAATTTCGCTGGGACTTGCGGC
GACTTCATCGTGGTCTCTGTCATTATGCGCTCCAACGCATAACTTTGCGCCAGAAGATAGATAGAATGGTGTAAGAAACT
GTAATATATATAATGAACTTCGGCGAGTCTGTGGAGTTTTTGTTGCATTAGAGAGCCAAGAGGTCGGACGTCCTCACGTA
GCCCGAGACGGGCAGGGCGATGGCGACTGAACGGGCTCCATATCACTTTGAGCTTTTATGCTTTCGACTCCTCCAGGAGC
TGAACAACCTTGTTCCCGGCAAAGCCCACTGCGTCATGGAGCTCACGGTCTACATTCATGACTGACTAACCGTAAACTGC
\end{lstlisting}

\subsection*{Output}
The output of the \texttt{gto\char`_genomic\char`_count\char`_bases} program is report which describes the number of each base in the file, and the file type.\\
Using the input above, an output example for this is the following:
\begin{lstlisting}
File type        : DNA
Number of bases  : 480
Number of a/A    : 114
Number of c/C    : 116
Number of g/G    : 120
Number of t/T    : 130
Number of n/N    : 0
Number of others : 0
\end{lstlisting}