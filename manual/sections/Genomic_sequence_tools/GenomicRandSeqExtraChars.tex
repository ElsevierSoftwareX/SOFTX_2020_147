\section{Program gto\char`_genomic\char`_rand\char`_seq\char`_extra\char`_chars}
The \texttt{gto\char`_genomic\char`_rand\char`_seq\char`_extra\char`_chars} substitues in the DNA sequence the outside ACGT chars by random ACGT symbols. It works in sequence file formats.\\
For help type:
\begin{lstlisting}
./gto_genomic_rand_seq_extra_chars -h
\end{lstlisting}
In the following subsections, we explain the input and output paramters.

\subsection*{Input parameters}

The \texttt{gto\char`_genomic\char`_rand\char`_seq\char`_extra\char`_chars} program needs two streams for the computation,
namely the input and output standard. The input stream is a sequence file.\\
The attribution is given according to:
\begin{lstlisting}
Usage: ./gto_genomic_rand_seq_extra_chars [options] [[--] args]
   or: ./gto_genomic_rand_seq_extra_chars [options]

It substitues in the DNA sequence the outside ACGT chars by random ACGT symbols.
It works in sequence file formats


    -h, --help        show this help message and exit

Basic options
    < input.seq       Input sequence file (stdin)
    > output.seq      Output sequence file (stdout)

Example: ./gto_genomic_rand_seq_extra_chars < input.seq > output.seq
\end{lstlisting}
An example on such an input file is:
\begin{lstlisting}
ANAAGACGNNNTCCTGCTGCTGCTGCTCTCCGGGGCCACGGCCCTGGAGGGTCCACCGCTGCCCTGCTGCCATTGTCCCC
NNCCCCACCTAAGGAAAAGCAGCCTCCTGACTTTCCTCGCTTGGGCCGAGACAGCGAGCATATGCAGGAAGCGGCAGGAA
GTGGTTTGAGTGGACCTCCGGGCCCCNNNNNGGAGAGGAAGCTCGGGAGNGTNNNGGCCAGGCGGCAGNNNNCCAGTGCC
GCGAATCCGCGCGCCGGGACAGAATCTCCTGCAAAGCCCTGCAGGAACTTCTTCTGGAAGACCTTCTCCACCCCCCCAGC
TANNNNCTCACCCATGAATGCTCACGCAAGTTTAATTACAGACCTGAAACAAGATGCCATTGTCCCCCGGCCTCCTGCTG
CTGCTGCTCTCCGGGGCCACGGCCACCGCTGCCCTGCCCCTGGAGGGTGGCCCCACCGGCCGAGACAGCGAGCATATGCA
GGAAGCGGCAGGAATAAGNNNAAGCAGCCTCCTGACTTTCCTCGCTTGNNNNTTTGAGTGGACCTCCCAGGCCAGTGCCG
GGCCCCTCATAGGAGAGGAAGCTCGGGAGGTGGCCAGGCGGCAGGAAGGCGCACCCCCCCAGCAATCCGCGCGCCGGGAC
AGAATGCCCTGCAGGAACTTCTTCTGGAAGACCTTCTCCTCCTGCAAATAAAACCTCACCCATGAATGCTCACGCAAGTT
NNATTACNNNCCTGNN
\end{lstlisting}

\subsection*{Output}
The output of the \texttt{gto\char`_genomic\char`_rand\char`_seq\char`_extra\char`_chars} program is a sequence file.\\
An example, for the input, is:
\begin{lstlisting}
ATAAGACGGCTTCCTGCTGCTGCTGCTCTCCGGGGCCACGGCCCTGGAGGGTCCACCGCTGCCCTGCTGCCATTGTCCCC
CTCCCCACCTAAGGAAAAGCAGCCTCCTGACTTTCCTCGCTTGGGCCGAGACAGCGAGCATATGCAGGAAGCGGCAGGAA
GTGGTTTGAGTGGACCTCCGGGCCCCGACCGGGAGAGGAAGCTCGGGAGTGTGTTGGCCAGGCGGCAGGAGACCAGTGCC
GCGAATCCGCGCGCCGGGACAGAATCTCCTGCAAAGCCCTGCAGGAACTTCTTCTGGAAGACCTTCTCCACCCCCCCAGC
TAATATCTCACCCATGAATGCTCACGCAAGTTTAATTACAGACCTGAAACAAGATGCCATTGTCCCCCGGCCTCCTGCTG
CTGCTGCTCTCCGGGGCCACGGCCACCGCTGCCCTGCCCCTGGAGGGTGGCCCCACCGGCCGAGACAGCGAGCATATGCA
GGAAGCGGCAGGAATAAGCGGAAGCAGCCTCCTGACTTTCCTCGCTTGGTTTTTTGAGTGGACCTCCCAGGCCAGTGCCG
GGCCCCTCATAGGAGAGGAAGCTCGGGAGGTGGCCAGGCGGCAGGAAGGCGCACCCCCCCAGCAATCCGCGCGCCGGGAC
AGAATGCCCTGCAGGAACTTCTTCTGGAAGACCTTCTCCTCCTGCAAATAAAACCTCACCCATGAATGCTCACGCAAGTT
CGATTACGGCCCTGTC
\end{lstlisting}