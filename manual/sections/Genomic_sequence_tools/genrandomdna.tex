\section{Program goose-genrandomdna}
The \texttt{goose-genrandomdna} generates a synthetic DNA.\\
For help type:
\begin{lstlisting}
./goose-genrandomdna -h
\end{lstlisting}
In the following subsections, we explain the input and output paramters.

\subsection*{Input parameters}

The \texttt{goose-genrandomdna} program needs one stream for the computation,
namely the output standard.\\
The attribution is given according to:
\begin{lstlisting}
Usage: ./goose-genrandomdna [options] [[--] args]
   or: ./goose-genrandomdna [options]

It generates a synthetic DNA.

    -h, --help                show this help message and exit

Basic options
    > output.seq              Output synthetic DNA sequence (stdout)
    -s, --seed=<int>          Starting point to the random generator (Default 0)
    -n, --nSymbols=<int>      Number of symbols generated (Default 100)
    -f, --frequency=<str>     The frequency of each base. It should be represented 
    						  in the following format: <fa,fc,fg,ft>.

Example: ./goose-genrandomdna > output.seq
\end{lstlisting}

\subsection*{Output}
The output of the \texttt{goose-genrandomdna} program is a sequence group file whith the synthetic DNA.\\
Using the seed value as 1 and the number of symbols as 400, an example of an execution, is: 
\begin{lstlisting}
TCTTTACTCGCGCGTTGGAGAAATACAATAGTGCGGCTCTGTCTCCTTATGAAGTCAACAATTTCGCTGGGACTTGCGGC
TCTTTACTCGCGCGTTGGAGAAATACAATAGTGCGGCTCTGTCTCCTTATGAAGTCAACAATTTCGCTGGGACTTGCGGC
GACTTCATCGTGGTCTCTGTCATTATGCGCTCCAACGCATAACTTTGCGCCAGAAGATAGATAGAATGGTGTAAGAAACT
GTAATATATATAATGAACTTCGGCGAGTCTGTGGAGTTTTTGTTGCATTAGAGAGCCAAGAGGTCGGACGTCCTCACGTA
GCCCGAGACGGGCAGGGCGATGGCGACTGAACGGGCTCCATATCACTTTGAGCTTTTATGCTTTCGACTCCTCCAGGAGC
TGAACAACCTTGTTCCCGGCAAAGCCCACTGCGTCATGGAGCTCACGGTCTACATTCATGACTGACTAACCGTAAACTGC
\end{lstlisting}
